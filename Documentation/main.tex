
%You should submit:

    %Functional source code, in a public github repository,

    %A 4-page summary, which should focus on the technical part: the algorithms, their implementation, and their performance.

    %Material for the oral presentation (slides), which should focus on the process: what decisions were taken and why, which strengths/shortcomings were found, and a critical discussions of the results.

%https://en.wikibooks.org/wiki/LaTeX/Document_Structure
\documentclass[11pt,twoside,a4paper]{article}

\usepackage{graphicx}

\title{Person Detector}
\author{Martí Gelabert Gómez}
\date{\today}

\begin{document}
\maketitle

\section{Introduction}


\section{Gabor Filter}

%Think as Gabor as a Gaussian filter in 2D.


A band pass filter generated by a function of various parameters.

\begin{equation}
    filter(x,y;\sigma,\theta,\lambda,\gamma,\phi) = exp [ - \frac{x^2 - \gamma ^2 \cdot y^2}{2 \sigma^2} ] \cdot exp [ i (2 \pi \frac{x}{\lambda} + \phi) ] 
\end{equation}


% Large sigma on small feature you will miss
% Small sigma on ...

The parameters of ksize allows to select the size of our kernel filter, in case we are using a really big shape we will overlook details if the shapes are small. The same reasoning can be applyed with a small filter, may overlook shapes too big for it. Therefore, must be tested with different sizes to reach an idoneal spot, if your features are tiny or bigger, you have to take that in count.

A common practice is to generate a \textbf{bank}  of filters by changing the different parameters introduced previously.

If we are looking for \textbf{horizontal-like} features, applying an horizontal filter will allow us to maintain those characteristics and block the vertical ones and viceversa in the other cases.

\section{What form do the People have?}


\end{document}