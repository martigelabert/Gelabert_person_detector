
%You should submit:

    %Functional source code, in a public github repository,

    %A 4-page summary, which should focus on the technical part: the algorithms, their implementation, and their performance.

    %Material for the oral presentation (slides), which should focus on the process: what decisions were taken and why, which strengths/shortcomings were found, and a critical discussions of the results.

%https://en.wikibooks.org/wiki/LaTeX/Document_Structure
\documentclass[11pt,twoside,a4paper]{article}

\usepackage{graphicx}

\title{An implementation for person detection* : A quick review of how computer vision algorithm can perform withouth AI
}
\author{Martí Gelabert Gómez}
\date{\today}

\begin{document}
\maketitle
\tableofcontents


\section{Introduction}




\section{Method 1}
\begin{enumerate}
\item Import images as black and white.
\item Apply Histogram equalization to the images.
\item Generate an image from the averaging from all the images we have already with the Histogram equalization already applyed.
\item Substract the background to the images using the average image.
\item Apply a thresholding algorithm to binarize the image.
\item Apply a dilation operation into the binarized images to expand the whites.
\item Use a contour algorithm to extract the diferent regions containing persons.
\end{enumerate}

\section{Method 2}
\begin{enumerate}
\item Import images in color.
\item Generate an image from the averaging from all the images we have.
\item Substract the background to the images using the average image.
\item Apply a thresholding algorithm to binarize the image.
\item Apply a dilation operation into the binarized images to expand the whites.
\item Use a contour algorithm to extract the diferent regions containing persons.
\end{enumerate}



\section{Preprocessing}


% Check spell
\section{Background Removal} 
Background removal is a process that allows the programmer to remove the background from the image, this way, the result image with this process allows a better binarization of the image.




\subsubsection{Image Averaging}



\subsubsection{Image Substraction}
%In the areas where there is a person, at the time of applying the image substraction, we will be  



\section{Gabor Filter}

%Think as Gabor as a Gaussian filter in 2D.


A band pass filter generated by a function of various parameters.

\begin{equation}
    filter(x,y;\sigma,\theta,\lambda,\gamma,\phi) = exp [ - \frac{x^2 - \gamma ^2 \cdot y^2}{2 \sigma^2} ] \cdot exp [ i (2 \pi \frac{x}{\lambda} + \phi) ] 
\end{equation}

% Large sigma on small feature you will miss
% Small sigma on ...

The parameters of ksize allows to select the size of our kernel filter, in case we are using a really big shape we will overlook details if the shapes are small. The same reasoning can be applyed with a small filter, may overlook shapes too big for it. Therefore, must be tested with different sizes to reach an idoneal spot, if your features are tiny or bigger, you have to take that in count.

If we are looking for \textbf{horizontal-like} features, applying an horizontal filter will allow us to maintain those characteristics and block the vertical ones and viceversa in the other cases.


\section{What form do the People have?}

In general the shapes that a person can describe may suffer alot of distorsion depending of the angle where the frame was took, the person posture, etc. Not always will be a perfet pose to the researchers to easily identify if an object is a person or not. There is no key shape that we could use, but we can try use common sense. The images gathered, are related to people moving standing up, swiming or just taking sunbathing.


\end{document}
